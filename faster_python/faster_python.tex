\documentclass{article}
\usepackage{hyperref}
\usepackage{listings}
\usepackage{color}
\definecolor{light-gray}{gray}{0.95}
\lstset{
  basicstyle=\ttfamily,
  backgroundcolor=\color{light-gray}}
\begin{document}
\title{How to Write Faster Python}
\author{Matt Belisle}
\maketitle
\tableofcontents
\pagebreak
\section{Introduction}
This article is a summary of what Matt learned at PyCon 2013 about how
to make Python programs faster.  A digital version of this article can
be found here:

\bigskip\noindent
\url{https://github.com/matthewbelisle-wf/faster_python/blob/master/faster_python.pdf}
\bigskip

\noindent In order to get the most out of this article, open up an
IPython prompt yourself and play around with the examples below.
Sections \ref{sec:builtin_types} and \ref{sec:appropriate_types} can
make a slow program orders of magnitude faster.  Sections
\ref{sec:attr_lookups} and \ref{sec:scope} can make a slow
program fractionally faster.
\section{Use Built-in Types and Functions}\label{sec:builtin_types}
This seems obvious but worth mentioning.  Built-in data structures
like list, dict, complex, etc. have two advantages compared to
whatever we write.
\begin{enumerate}
  \item They were written in C.
  \item They were written by Guido.
\end{enumerate}
Before writing a class or a function make sure there isn't one already
built into Python.  There are a lot of built in functions that most of
us don't know about.
\section{Use the Appropriate Data Type}\label{sec:appropriate_types}
A common mistake for Python programmers is to use a list when we
should be using a tuple, set, deque, or a generator instead.
\begin{itemize}
  \item List\\\url{http://docs.python.org/2/library/functions.html#list}
  \item Tuple\\\url{http://docs.python.org/2/library/functions.html#tuple}
  \item Set\\\url{http://docs.python.org/2/library/functions.html#set}
  \item Deque\\\url{http://docs.python.org/2/library/collections.html#collections.deque}
  \item Generator\\\url{http://docs.python.org/2/reference/simple_stmts.html#yield}
\end{itemize}
\subsection{Set vs List}
For inclusion testing, use a set which is $O(1)$ time complexity
instead of a list which is $O(n)$.

\bigskip
\lstinputlisting{includes/set.txt}
\bigskip
\subsection{Deque vs List}
For prepending/appending, use a deque which is $O(1)$ time complexity
instead of a list which is $O(n)$.

\bigskip
\lstinputlisting{includes/deque.txt}
\bigskip
\subsection{Tuple vs List}
Use a tuple when it's ok for the object to be immutable since a tuple
takes up less memory.

\bigskip
\lstinputlisting{includes/tuple.txt}
\bigskip
\subsection{Generator vs List}
For iterables where you only need to loop through one variable at a
time you should use a generator like xrange(), which is $O(1)$ space
complexity instead of a list like range() which is $O(n)$ space
complexity.

\bigskip
\lstinputlisting{includes/generator.txt}
\bigskip
\subsection{Quiz}
Instead of a list, which data type would make this intersect call 200
times faster?\footnote{Answer: A set.  Extra credit if you noticed
  that using the built-in intersect() method would make it 800 times
  faster.}

\bigskip
\lstinputlisting{includes/quiz.txt}
\bigskip
\section{Use Attribute Lookups Sparingly}\label{sec:attr_lookups}
Accessing an attribute on an object performs a hash table lookup
behind the scenes, so it is best to avoid using attributes inside a
loop.

\bigskip
\lstinputlisting{includes/sqrt.txt}
\bigskip
\section{Use Local Scope when Possible}\label{sec:scope}
When the python interpreter does a variable lookup, it searches the
local scope first.  If it can't find it there, it searches the global
scope, and if it can't find it there it searches the built-in scope
before throwing a NameError.

\bigskip\noindent
\url{http://docs.python.org/2/tutorial/classes.html#python-scopes-and-namespaces}
\bigskip

\noindent This means that local variable lookups are a little faster
than global variable lookups.

\bigskip
\lstinputlisting{includes/scope.txt}
\bigskip
\section{Test for Hotspots}
There are several tools available for locating inefficient parts of a
program.  Use whichever one you like best.
\subsection{Time Complexity}
\begin{itemize}
  \item timeit - Times one line snippets
    \\\url{http://docs.python.org/2/library/timeit.html}
  \item cProfile - Collects stats about time spent in each function
    \\\url{http://docs.python.org/2/library/profile.html#module-cProfile}
  \item kernprof.py - Collects stats about time spent on each line
    \\\url{https://pypi.python.org/pypi/line_profiler}
\end{itemize}
\subsection{Memory Complexity}
\begin{itemize}
  \item heapy - Takes memory measuremeants inside Python on any OS
    \\\url{https://pypi.python.org/pypi/guppy/}
  \item valgrind - Takes memory measurements for any program running on Unix
    \\\url{http://valgrind.org/}
\end{itemize}
\end{document}
